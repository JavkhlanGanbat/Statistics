% Options for packages loaded elsewhere
% Options for packages loaded elsewhere
\PassOptionsToPackage{unicode}{hyperref}
\PassOptionsToPackage{hyphens}{url}
\PassOptionsToPackage{dvipsnames,svgnames,x11names}{xcolor}
%
\documentclass[
]{article}
\usepackage{xcolor}
\usepackage[margin=1in]{geometry}
\usepackage{amsmath,amssymb}
\setcounter{secnumdepth}{5}
\usepackage{iftex}
\ifPDFTeX
  \usepackage[T1]{fontenc}
  \usepackage[utf8]{inputenc}
  \usepackage{textcomp} % provide euro and other symbols
\else % if luatex or xetex
  \usepackage{unicode-math} % this also loads fontspec
  \defaultfontfeatures{Scale=MatchLowercase}
  \defaultfontfeatures[\rmfamily]{Ligatures=TeX,Scale=1}
\fi
\usepackage{lmodern}
\ifPDFTeX\else
  % xetex/luatex font selection
\fi
% Use upquote if available, for straight quotes in verbatim environments
\IfFileExists{upquote.sty}{\usepackage{upquote}}{}
\IfFileExists{microtype.sty}{% use microtype if available
  \usepackage[]{microtype}
  \UseMicrotypeSet[protrusion]{basicmath} % disable protrusion for tt fonts
}{}
\makeatletter
\@ifundefined{KOMAClassName}{% if non-KOMA class
  \IfFileExists{parskip.sty}{%
    \usepackage{parskip}
  }{% else
    \setlength{\parindent}{0pt}
    \setlength{\parskip}{6pt plus 2pt minus 1pt}}
}{% if KOMA class
  \KOMAoptions{parskip=half}}
\makeatother
% Make \paragraph and \subparagraph free-standing
\makeatletter
\ifx\paragraph\undefined\else
  \let\oldparagraph\paragraph
  \renewcommand{\paragraph}{
    \@ifstar
      \xxxParagraphStar
      \xxxParagraphNoStar
  }
  \newcommand{\xxxParagraphStar}[1]{\oldparagraph*{#1}\mbox{}}
  \newcommand{\xxxParagraphNoStar}[1]{\oldparagraph{#1}\mbox{}}
\fi
\ifx\subparagraph\undefined\else
  \let\oldsubparagraph\subparagraph
  \renewcommand{\subparagraph}{
    \@ifstar
      \xxxSubParagraphStar
      \xxxSubParagraphNoStar
  }
  \newcommand{\xxxSubParagraphStar}[1]{\oldsubparagraph*{#1}\mbox{}}
  \newcommand{\xxxSubParagraphNoStar}[1]{\oldsubparagraph{#1}\mbox{}}
\fi
\makeatother


\usepackage{longtable,booktabs,array}
\usepackage{calc} % for calculating minipage widths
% Correct order of tables after \paragraph or \subparagraph
\usepackage{etoolbox}
\makeatletter
\patchcmd\longtable{\par}{\if@noskipsec\mbox{}\fi\par}{}{}
\makeatother
% Allow footnotes in longtable head/foot
\IfFileExists{footnotehyper.sty}{\usepackage{footnotehyper}}{\usepackage{footnote}}
\makesavenoteenv{longtable}
\usepackage{graphicx}
\makeatletter
\newsavebox\pandoc@box
\newcommand*\pandocbounded[1]{% scales image to fit in text height/width
  \sbox\pandoc@box{#1}%
  \Gscale@div\@tempa{\textheight}{\dimexpr\ht\pandoc@box+\dp\pandoc@box\relax}%
  \Gscale@div\@tempb{\linewidth}{\wd\pandoc@box}%
  \ifdim\@tempb\p@<\@tempa\p@\let\@tempa\@tempb\fi% select the smaller of both
  \ifdim\@tempa\p@<\p@\scalebox{\@tempa}{\usebox\pandoc@box}%
  \else\usebox{\pandoc@box}%
  \fi%
}
% Set default figure placement to htbp
\def\fps@figure{htbp}
\makeatother


% definitions for citeproc citations
\NewDocumentCommand\citeproctext{}{}
\NewDocumentCommand\citeproc{mm}{%
  \begingroup\def\citeproctext{#2}\cite{#1}\endgroup}
\makeatletter
 % allow citations to break across lines
 \let\@cite@ofmt\@firstofone
 % avoid brackets around text for \cite:
 \def\@biblabel#1{}
 \def\@cite#1#2{{#1\if@tempswa , #2\fi}}
\makeatother
\newlength{\cslhangindent}
\setlength{\cslhangindent}{1.5em}
\newlength{\csllabelwidth}
\setlength{\csllabelwidth}{3em}
\newenvironment{CSLReferences}[2] % #1 hanging-indent, #2 entry-spacing
 {\begin{list}{}{%
  \setlength{\itemindent}{0pt}
  \setlength{\leftmargin}{0pt}
  \setlength{\parsep}{0pt}
  % turn on hanging indent if param 1 is 1
  \ifodd #1
   \setlength{\leftmargin}{\cslhangindent}
   \setlength{\itemindent}{-1\cslhangindent}
  \fi
  % set entry spacing
  \setlength{\itemsep}{#2\baselineskip}}}
 {\end{list}}
\usepackage{calc}
\newcommand{\CSLBlock}[1]{\hfill\break\parbox[t]{\linewidth}{\strut\ignorespaces#1\strut}}
\newcommand{\CSLLeftMargin}[1]{\parbox[t]{\csllabelwidth}{\strut#1\strut}}
\newcommand{\CSLRightInline}[1]{\parbox[t]{\linewidth - \csllabelwidth}{\strut#1\strut}}
\newcommand{\CSLIndent}[1]{\hspace{\cslhangindent}#1}



\setlength{\emergencystretch}{3em} % prevent overfull lines

\providecommand{\tightlist}{%
  \setlength{\itemsep}{0pt}\setlength{\parskip}{0pt}}



 


\makeatletter
\@ifpackageloaded{caption}{}{\usepackage{caption}}
\AtBeginDocument{%
\ifdefined\contentsname
  \renewcommand*\contentsname{Table of contents}
\else
  \newcommand\contentsname{Table of contents}
\fi
\ifdefined\listfigurename
  \renewcommand*\listfigurename{List of Figures}
\else
  \newcommand\listfigurename{List of Figures}
\fi
\ifdefined\listtablename
  \renewcommand*\listtablename{List of Tables}
\else
  \newcommand\listtablename{List of Tables}
\fi
\ifdefined\figurename
  \renewcommand*\figurename{Figure}
\else
  \newcommand\figurename{Figure}
\fi
\ifdefined\tablename
  \renewcommand*\tablename{Table}
\else
  \newcommand\tablename{Table}
\fi
}
\@ifpackageloaded{float}{}{\usepackage{float}}
\floatstyle{ruled}
\@ifundefined{c@chapter}{\newfloat{codelisting}{h}{lop}}{\newfloat{codelisting}{h}{lop}[chapter]}
\floatname{codelisting}{Listing}
\newcommand*\listoflistings{\listof{codelisting}{List of Listings}}
\makeatother
\makeatletter
\makeatother
\makeatletter
\@ifpackageloaded{caption}{}{\usepackage{caption}}
\@ifpackageloaded{subcaption}{}{\usepackage{subcaption}}
\makeatother
\usepackage{bookmark}
\IfFileExists{xurl.sty}{\usepackage{xurl}}{} % add URL line breaks if available
\urlstyle{same}
\hypersetup{
  pdftitle={Statistical Analysis of Income Prediction},
  pdfauthor={Javkhlan},
  colorlinks=true,
  linkcolor={blue},
  filecolor={Maroon},
  citecolor={Blue},
  urlcolor={Blue},
  pdfcreator={LaTeX via pandoc}}


\title{Statistical Analysis of Income Prediction}
\usepackage{etoolbox}
\makeatletter
\providecommand{\subtitle}[1]{% add subtitle to \maketitle
  \apptocmd{\@title}{\par {\large #1 \par}}{}{}
}
\makeatother
\subtitle{Logistic Regression and Probabilistic Modeling}
\author{Javkhlan}
\date{2025-12-01}
\begin{document}
\maketitle

\renewcommand*\contentsname{Table of contents}
{
\hypersetup{linkcolor=}
\setcounter{tocdepth}{3}
\tableofcontents
}

\section{Introduction}\label{introduction}

This report analyzes the Adult Income dataset to predict whether an
individual earns more than \$50,000 annually. We employ \textbf{Logistic
Regression}, a fundamental statistical method for binary classification
that models the probability of class membership.

The focus is on the probabilistic interpretation of the model, the
derivation of the loss function via Maximum Likelihood Estimation (MLE),
and the evaluation of the model using statistical metrics.

\subsection{Educational Context}\label{educational-context}

Beyond the application, this project emphasizes the educational value of
implementing statistical algorithms from first principles. Rather than
relying solely on ``black-box'' implementations from libraries like
Scikit-Learn, we utilize a custom implementation of Logistic Regression.
This allows for a transparent examination of the optimization
process---specifically how Gradient Descent navigates the loss landscape
to find optimal coefficients.

\section{Mathematical Framework}\label{mathematical-framework}

\subsection{The Logistic Model}\label{the-logistic-model}

We model the probability that the target variable \(Y\) takes the value
1 (income \textgreater{} 50K) given input features \(X=x\) using the
sigmoid function \(\sigma(z)\):

\[
P(Y=1|X=x) = \sigma(w^T x + b) = \frac{1}{1 + e^{-(w^T x + b)}}
\]

where \(w\) are the weights and \(b\) is the bias. The linear
combination \(z = w^T x + b\) represents the log-odds (logit).

\subsection{Maximum Likelihood
Estimation}\label{maximum-likelihood-estimation}

To estimate parameters \(w\) and \(b\), we maximize the likelihood of
the observed data. Assuming independent and identically distributed
(i.i.d.) samples, the likelihood \(L\) is:

\[
L(w, b) = \prod_{i=1}^{m} P(y^{(i)} | x^{(i)})^{y^{(i)}} (1 - P(y^{(i)} | x^{(i)}))^{1-y^{(i)}}
\]

Taking the negative logarithm gives us the \textbf{Binary Cross-Entropy}
loss function, which we minimize:

\[
J(w, b) = -\frac{1}{m} \sum_{i=1}^{m} \left[ y^{(i)} \log(\hat{y}^{(i)}) + (1-y^{(i)}) \log(1-\hat{y}^{(i)}) \right]
\]

Unlike Mean Squared Error (MSE), which is non-convex for logistic
regression, the Log-Loss function is convex, guaranteeing that Gradient
Descent will converge to the global minimum.

\subsection{Regularization}\label{regularization}

To prevent overfitting, we introduce an \(L_2\) regularization term
(Ridge), which corresponds to placing a Gaussian prior on the weights
\(w \sim \mathcal{N}(0, \tau^2)\). The objective function becomes:

\[
J_{reg}(w, b) = J(w, b) + \frac{\lambda}{2m} ||w||^2
\]

\section{Data and Methodology}\label{data-and-methodology}

We utilize the Adult Income dataset. The data is split into training
(80\%) and validation (20\%) sets.

\subsection{Data Characteristics and
Preprocessing}\label{data-characteristics-and-preprocessing}

The dataset contains a mix of numerical (e.g., Age, Capital Gain) and
categorical (e.g., Education, Occupation) features. A critical challenge
in this dataset is \textbf{class imbalance}: approximately 76\% of
individuals earn \(\le 50\)K, while only 24\% earn \(>50\)K. This
imbalance implies that a naive model predicting the majority class for
every instance would achieve 76\% accuracy but have zero predictive
power for the target class.

To prepare the data for our gradient-based optimization: 1.
\textbf{One-Hot Encoding}: Categorical variables are transformed into
binary vectors. 2. \textbf{Standard Scaling}: Numerical features are
normalized to have mean 0 and variance 1. This is crucial for Gradient
Descent, as it ensures the loss landscape is symmetric, preventing the
optimizer from oscillating or converging slowly.

\section{Results}\label{results}

\subsection{Performance Metrics}\label{performance-metrics}

\begin{longtable}[]{@{}ll@{}}
\toprule\noalign{}
Metric & Value \\
\midrule\noalign{}
\endhead
\bottomrule\noalign{}
\endlastfoot
Accuracy & 0.85 \\
Precision & 0.79 \\
Recall & 0.72 \\
F1 Score & 0.75 \\
\end{longtable}

The F1 Score, which balances precision and recall, is particularly
important given the class imbalance. A naive accuracy measure would be
misleadingly high due to the 76\% majority class.

\subsection{Confusion Matrix}\label{confusion-matrix}

The confusion matrix is displayed using a plot.

\subsection{ROC Curve}\label{roc-curve}

The ROC curve demonstrates the model's ability to distinguish between
classes. The area under the curve (AUC) provides a single metric for
model performance across all classification thresholds.

\subsection{Feature Importance}\label{feature-importance}

The coefficients of the model indicate the relative importance of each
feature. A higher absolute value of a coefficient implies a greater
impact on the model's predictions. Features like \texttt{age},
\texttt{education-num}, and \texttt{hours-per-week} show significant
importance.

\section{Discussion}\label{discussion}

\begin{itemize}
\tightlist
\item
  Quarto enables reproducible, documented analysis with code and
  narrative.
\item
  Equations like \textbf{?@eq-logreg} are first-class citizens alongside
  figures and tables.
\item
  Use citations such as Roback and Legler
  (\citeproc{ref-roback2021beyond}{2021}) to anchor methods in
  literature.
\end{itemize}

\subsection{Statistical
Interpretation}\label{statistical-interpretation}

The default threshold for classifying an instance as income
\textgreater50K is 0.5. However, this can be adjusted depending on the
desired balance between precision and recall. For instance, in scenarios
where false negatives are costly, lowering the threshold might be
beneficial.

\subsection{Limitations and Future
Work}\label{limitations-and-future-work}

\begin{itemize}
\tightlist
\item
  The assumption of linearity between the log-odds of the outcome and
  the predictors may not hold in all cases.
\item
  The model does not account for potential interactions between
  features.
\item
  Future work could explore non-linear models or ensemble methods for
  comparison.
\end{itemize}

\section{Conclusion}\label{conclusion}

This template can be cloned for new reports. Replace text, update
references, and embed your analysis code.

\section{Appendix}\label{appendix}

\subsection{Additional Math}\label{additional-math}

A simple linear model:

\begin{equation}\phantomsection\label{eq-ols}{
y = X\beta + \varepsilon
}\end{equation}

Refer to Equation Equation~\ref{eq-ols} in text.

\subsection{Re-usable Blocks}\label{re-usable-blocks}

\begin{itemize}
\tightlist
\item
  Use sections and sub-sections to structure content.
\item
  Add callouts, code-folding, and filters as needed.
\item
  Keep references in references.bib and cite with \texttt{@key}.
\end{itemize}

\section*{References}\label{references}
\addcontentsline{toc}{section}{References}

\phantomsection\label{refs}
\begin{CSLReferences}{1}{0}
\bibitem[\citeproctext]{ref-roback2021beyond}
Roback, Paul, and Julie Legler. 2021. \emph{Beyond Multiple Linear
Regression: Applied Generalized Linear Models and Multilevel Models in
r}. Chapman; Hall/CRC.
\url{https://bookdown.org/roback/bookdown-BeyondMLR/ch-MLRreview.html}.

\end{CSLReferences}




\end{document}
